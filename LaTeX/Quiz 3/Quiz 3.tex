\documentclass[11pt]{article} 
\usepackage[english]{babel}
\usepackage[utf8]{inputenc}
\usepackage[margin=0.5in]{geometry}
\usepackage{amsmath}
\usepackage{amsthm}
\usepackage{amsfonts}
\usepackage{amssymb}
\usepackage[usenames,dvipsnames]{xcolor}
\usepackage{graphicx}
\usepackage[siunitx]{circuitikz}
\usepackage{tikz}
\usepackage[colorinlistoftodos, color=orange!50]{todonotes}
\usepackage{hyperref}
\usepackage[numbers, square]{natbib}
\usepackage{fancybox}
\usepackage{epsfig}
\usepackage{soul}
\usepackage[framemethod=tikz]{mdframed}
\usepackage[shortlabels]{enumitem}
\usepackage[version=4]{mhchem}
\usepackage{multicol}

\usepackage{mathtools}
\usepackage{comment}
\usepackage{enumitem}
\usepackage[utf8]{inputenc}
\usepackage[linesnumbered,ruled,vlined]{algorithm2e}
\usepackage{listings}
\usepackage{color}
\usepackage[numbers]{natbib}
\usepackage{subfiles}
\usepackage{tkz-berge}


\newtheorem{prop}{Proposition}[section]
\newtheorem{thm}{Theorem}[section]
\newtheorem{lemma}{Lemma}[section]
\newtheorem{cor}{Corollary}[prop]

\theoremstyle{definition}
\newtheorem{definition}{Definition}

\theoremstyle{definition}
\newtheorem{required}{Problem}
\newtheorem*{requiredHC}{Problem HC}


\theoremstyle{definition}
\newtheorem{ex}{Example}


\setlength{\marginparwidth}{3.4cm}
%#########################################################

%To use symbols for footnotes
\renewcommand*{\thefootnote}{\fnsymbol{footnote}}
%To change footnotes back to numbers uncomment the following line
%\renewcommand*{\thefootnote}{\arabic{footnote}}

% Enable this command to adjust line spacing for inline math equations.
% \everymath{\displaystyle}

% _______ _____ _______ _      ______ 
%|__   __|_   _|__   __| |    |  ____|
%   | |    | |    | |  | |    | |__   
%   | |    | |    | |  | |    |  __|  
%   | |   _| |_   | |  | |____| |____ 
%   |_|  |_____|  |_|  |______|______|
%%%%%%%%%%%%%%%%%%%%%%%%%%%%%%%%%%%%%%%

\title{
\normalfont \normalsize 
\textsc{CSCI 3104 Spring 2023 \\ 
Instructor: Prof. Ryan Layer and Chandra Kanth Nagesh} \\
[10pt] 
\rule{\linewidth}{0.5pt} \\[6pt] 
\huge Quiz 3 Standard 3 - Dijkstra's Algorithm \\
\rule{\linewidth}{2pt}  \\[10pt]
}
%\author{Your Name}
\date{}

\begin{document}

\maketitle


%%%%%%%%%%%%%%%%%%%%%%%%%
%%%%%%%%%%%%%%%%%%%%%%%%%%
%%%%%%%%%%FILL IN YOUR NAME%%%%%%%
%%%%%%%%%%AND STUDENT ID%%%%%%%%
%%%%%%%%%%%%%%%%%%%%%%%%%%
\noindent
Due Date \dotfill Thursday Sep 22, 8pm MT \\
Name \dotfill \textbf{Blake Raphael} \\
Student ID \dotfill \textbf{109752312} \\
Quiz Code (enter in Canvas to get access to the LaTeX template) \dotfill \textbf{YUIOP}


\tableofcontents

\section*{Instructions}
\addcontentsline{toc}{section}{Instructions}
 \begin{itemize}
	\item You may either type your work using this template, or you may handwrite your work and embed it as an image in this template. \textbf{If you choose to handwrite your work, the image must be legible, and oriented so that we do not have to rotate our screens to grade your work.} We have included some helpful LaTeX commands for including and rotating images commented out near the end of the LaTeX template.
	\item You should submit your work through the \textbf{class Gradescope page} only. Please submit one PDF file, compiled using this \LaTeX \ template.
	\item You may not need a full page for your solutions; pagebreaks are there to help Gradescope automatically find where each problem is. Even if you do not attempt every problem, please submit this document with no fewer pages than the blank template (or Gradescope has issues with it).

	\item You \textbf{may not collaborate with other students}. \textbf{Copying from any source is an Honor Code violation. Furthermore, all submissions must be in your own words and reflect your understanding of the material.} If there is any confusion about this policy, it is your responsibility to clarify before the due date. 

	\item Posting to \textbf{any} service including, but not limited to Chegg, Discord, Reddit, StackExchange, etc., for help on an assignment is a violation of the Honor Code.

	\item You \textbf{must} virtually sign the Honor Code (see Section \ref{HonorCode}). Failure to do so will result in your assignment not being graded.
\end{itemize}


\newpage
\section*{Honor Code (Make Sure to Virtually Sign)} \label{HonorCode}
\addcontentsline{toc}{section}{Honor Code (Make Sure to Virtually Sign)}
\hypertarget{HonorCode}{}

\begin{requiredHC}
\begin{itemize}
\item My submission is in my own words and reflects my understanding of the material.
\item Any collaborations and external sources have been clearly cited in this document.
\item I have not posted to external services including, but not limited to Chegg, Reddit, StackExchange, etc.
\item I have neither copied nor provided others solutions they can copy.
\end{itemize}

%\noindent In the specified region below, clearly indicate that you have upheld the Honor Code. Then type your name. 
\end{requiredHC}

\begin{proof}[Agreed (I agree to the above, Blake Raphael).]
%% Typing "I agree to the above," followed by your name is sufficient.
\end{proof}



\newpage
\setcounter{section}{2}
\section{Standard 3 - Dijkstra's Algorithm}

\setcounter{required}{2}
\begin{required} \label{Dijkstra1}
Consider the weighted graph $G(V, E, w)$ pictured below. Work through Dijkstra's algorithm on the following graph, using the source vertex $S$. 
\begin{itemize}
\item Clearly include the contents of the priority queue \emph{in priority order}, as well as the distance from $S$ to each vertex at each iteration.
\item If you use a table to store the distances, clearly label the keys according to the vertex names rather than numeric indices (i.e., \texttt{dist[`B']} is more descriptive than \texttt{dist[`1']}). 
\item You do \textbf{not} need to draw the graph at each iteration, though you are welcome to do so. [This may be helpful scratch work, which you do not need to include.]
\end{itemize}

\definecolor {processblue}{cmyk}{0.96,0,0,0}
\begin{center}
	\begin {tikzpicture}[-latex ,auto ,node distance =2 cm and 3cm ,on grid ,
	semithick ,
	state/.style ={ circle ,top color =white , bottom color = processblue!20 ,
	draw,processblue , text=blue , minimum width =1 cm}, scale=0.5]

	\node[state] (S) {$S$};
	\node[state] (B) [above right = of S] {$B$};
	\node[state] (C) [below right = of S] {$C$};
	\node[state] (F) [below right = of B] {$F$};
	\node[state] (D) [above right = of F] {$D$};
	\node[state] (E) [below right = of F] {$E$};
	
	\path (S) edge node[above] {$4$} (B);
	\path (S) edge node[above] {$9$} (C);
	\path (B) edge node[above] {$5$} (F);
	\path (B) edge node[above] {$2$} (D);
	\path (F) edge node[above] {$2$} (S);
	\path (E) edge node[above] {$1$} (C);
	\path (D) edge node[above] {$2$} (F);
	\path (F) edge node[above] {$1$} (E);
	\path (F) edge node[above] {$3$} (C);
	
	\end{tikzpicture}  
\end{center}

\end{required}

\begin{proof}[Answer]
\indent The following block will outline the priority queue step-by-step (distances order corresponds to the priority queue):
\begin{table}[h!]
	\begin{center}
		\caption{Priority Queue Contents}
		\label{tab:table1}
		\begin{tabular}{l|c|r} % <-- Alignments: 1st column left, 2nd middle and 3rd right, with vertical lines in between
			\textbf{Step} & \textbf{Queue Contents} & \textbf{Distance}\\
			\hline
			0 & S & N/A\\
			1 & B & 4\\
			2 & B, C & 4, 9\\
			3 & D, C & 6, 9\\
			4 & D, C, F & 6, 9, 9\\
			5 & F, C & 8, 9\\
			6 & C, E & 9, 9\\
			7 & E & 9\\
			8 & All Nodes Explored & All Nodes Explored\\
		\end{tabular}
	\end{center}
\end{table}

\indent Final Distances:\\
\indent $S = 0$\\
\indent $B = 4$\\
\indent $D = 6$\\
\indent $F = 8$\\
\indent $C = 9$\\
\indent $E = 9$\\
\end{proof}




%%%%%%%%%%%%%%%%%%%%%%%%%%%%%%%%%%%%%%%%%%%%%%%%%%
\end{document} % NOTHING AFTER THIS LINE IS PART OF THE DOCUMENT



