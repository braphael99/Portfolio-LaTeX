\documentclass[11pt]{article}
\usepackage[english]{babel}
\usepackage[utf8]{inputenc}
\usepackage[margin=0.5in]{geometry}
\usepackage{amsmath}
\usepackage{amsthm}
\usepackage{amsfonts}
\usepackage{amssymb}
\usepackage[usenames,dvipsnames]{xcolor}
\usepackage{graphicx}
\usepackage[siunitx]{circuitikz}
\usepackage{tikz}
\usepackage[colorinlistoftodos, color=orange!50]{todonotes}
\usepackage{hyperref}
\usepackage[numbers, square]{natbib}
\usepackage{fancybox}
\usepackage{epsfig}
\usepackage{soul}
\usepackage[framemethod=tikz]{mdframed}
\usetikzlibrary{positioning, automata, backgrounds}
\usepackage[shortlabels]{enumitem}
\usepackage[version=4]{mhchem}
\usepackage{multicol}
\usepackage{forest}
\usepackage{mathtools}
\usepackage{comment}
\usepackage{enumitem}
\usepackage[utf8]{inputenc}
\usepackage[linesnumbered,ruled,vlined]{algorithm2e}
\usepackage{listings}
\usepackage{color}
\usepackage[numbers]{natbib}
\usepackage{subfiles}
\usepackage{tkz-berge}


\newcommand{\interval}[4]{\draw (#2, #1) -- (#3, #1); % Usage: \interval{height}{start}{end}{label}
\draw (#2, #1-0.11) -- (#2, #1+0.11); % draw left whisker
\draw (#3, #1-0.11) -- (#3, #1+0.11); % draw right whisker
\node[] at (#2-0.25, #1) {#4};
}

\newtheorem{prop}{Proposition}[section]
\newtheorem{thm}{Theorem}[section]
\newtheorem{lemma}{Lemma}[section]
\newtheorem{cor}{Corollary}[prop]

\theoremstyle{definition}
\newtheorem{definition}{Definition}

\theoremstyle{definition}
\newtheorem{required}{Problem}

\theoremstyle{definition}
\newtheorem{ex}{Example}


\setlength{\marginparwidth}{3.4cm}
%#########################################################

%To use symbols for footnotes
\renewcommand*{\thefootnote}{\fnsymbol{footnote}}
%To change footnotes back to numbers uncomment the following line
%\renewcommand*{\thefootnote}{\arabic{footnote}}

% Enable this command to adjust line spacing for inline math equations.
% \everymath{\displaystyle}

% _______ _____ _______ _      ______ 
%|__   __|_   _|__   __| |    |  ____|
%   | |    | |    | |  | |    | |__   
%   | |    | |    | |  | |    |  __|  
%   | |   _| |_   | |  | |____| |____ 
%   |_|  |_____|  |_|  |______|______|
%%%%%%%%%%%%%%%%%%%%%%%%%%%%%%%%%%%%%%%

\title{
\normalfont \normalsize 
\textsc{CSCI 3104 Spring 2023 \\ 
Instructors: Ryan Layer and Chandra Kanth Nagesh} \\
[10pt] 
\rule{\linewidth}{0.5pt} \\[6pt] 
\huge Homework 9 \\
\rule{\linewidth}{2pt}  \\[10pt]
}
%\author{}
\date{}

\begin{document}

\definecolor {processblue}{cmyk}{0.96,0,0,0}
\definecolor{processred}{rgb}{200, 0, 0}
\definecolor{processgreen}{rgb}{0, 255, 0}
\DeclareGraphicsExtensions{.png}
\DeclareGraphicsExtensions{.gif}
\DeclareGraphicsExtensions{.jpg}

\maketitle


%%%%%%%%%%%%%%%%%%%%%%%%%
%%%%%%%%%%%%%%%%%%%%%%%%%%
%%%%%%%%%%FILL IN YOUR NAME%%%%%%%
%%%%%%%%%%AND STUDENT ID%%%%%%%%
%%%%%%%%%%%%%%%%%%%%%%%%%%
\noindent
Due Date \dotfill February 23, 2023 \\
Name \dotfill \textbf{Blake Raphael} \\
Student ID \dotfill \textbf{109752312} \\
Collaborators \dotfill \textbf{Alex Barry, Brody Cyphers, and Ben Kohav}

\tableofcontents

\section{Instructions}
 \begin{itemize}
	\item The solutions \textbf{should be typed}, using proper mathematical notation. We cannot accept hand-written solutions. \href{http://ece.uprm.edu/~caceros/latex/introduction.pdf}{Here's a short intro to \LaTeX.}
	\item You should submit your work through the \textbf{class Gradescope page} only (linked from Canvas). Please submit one PDF file, compiled using this \LaTeX \ template.
	\item You may not need a full page for your solutions; pagebreaks are there to help Gradescope automatically find where each problem is. Even if you do not attempt every problem, please submit this document with no fewer pages than the blank template (or Gradescope has issues with it).

	\item You are welcome and encouraged to collaborate with your classmates, as well as consult outside resources. You must \textbf{cite your sources in this document.} \textbf{Copying from any source is an Honor Code violation. Furthermore, all submissions must be in your own words and reflect your understanding of the material.} If there is any confusion about this policy, it is your responsibility to clarify before the due date. 

	\item Posting to \textbf{any} service including, but not limited to Chegg, Reddit, StackExchange, etc., for help on an assignment is a violation of the Honor Code.

	\item You \textbf{must} virtually sign the Honor Code (see Section \ref{HonorCode}). Failure to do so will result in your assignment not being graded.
\end{itemize}


\section{Honor Code (Make Sure to Virtually Sign)} \label{HonorCode}

\begin{itemize}
\item My submission is in my own words and reflects my understanding of the material.
\item Any collaborations and external sources have been clearly cited in this document.
\item I have not posted to external services including, but not limited to Chegg, Reddit, StackExchange, etc.
\item I have neither copied nor provided others solutions they can copy.
\end{itemize}

%\noindent In the specified region below, clearly indicate that you have upheld the Honor Code. Then type your name. 

\begin{proof}[Agreed (I agree to the above, Blake Raphael).]
%% Typing "I agree to the above," followed by your name is sufficient.
\end{proof}

\newpage
\section{Standard 9- Network Flows: Terminology}
%\setcounter{subsection}{1}
\subsection{Problem \ref{Flow_Term}}
\begin{required} \label{Flow_Term}
Consider the following flow network, with the following flow configuration $f$ as indicated below. 
\begin{center}
	\begin {tikzpicture}[-latex ,auto ,node distance =2 cm and 3cm ,on grid ,
	semithick ,
	state/.style ={ circle ,top color =white , bottom color = processblue!20 ,
	draw,processblue , text=blue , minimum width =1 cm}]

	\node[state] (A) {$s$};
	\node[state] (B) [above right = of A] {$B$};
	\node[state] (C) [below right = of A] {$C$};
	\node[state] (D) [right = of B] {$D$};
	\node[state] (E) [right = of C] {$E$};
	\node[state] (F) [below right = of D] {$t$};
	
	\path (A) edge node[above] {$1 / 6$} (B);
	\path (A) edge node[above] {$3 / 7$} (C);
	\path (B) edge node[left] {$0 / 2$} (C);
	\path (B) edge node[above] {$3 / 3$} (D);
	\path (E) edge node[right] {$2 /  5$} (B);
	\path (C) edge node[above] {$3 / 10$} (E);
	\path (E) edge node[right] {$0 / 4$} (D);
	\path (D) edge node[above] {$3 / 9$} (F);
	\path (E) edge node[above] {$1 / 1$} (F);
	
	\end{tikzpicture}  
\end{center}

\noindent Do the following.
\begin{enumerate}[label=(\alph*)]
\subsubsection{Problem 1\ref{a} (0.5 points)}
\item \label{a}

    Given the current flow configuration $f$, what is the maximum \textit{additional} amount of flow that we can push across the edge $(B, D)$ from $B \to D$? Justify using 1-2 sentences.

\begin{proof}[Answer]
We cannot push anymore flow through this pipe because it has already reached its capacity. In the augmented flow graph, this edge is reversed, showing that we reached capacity.\\
\end{proof}


\vskip 10pt
\subsubsection{Problem 1\ref{b} (0.5 points)}
\item \label{b} Given the current flow configuration $f$, what is the maximum amount of flow that $B$ can push backwards to $E$? Do \textbf{not} consider whether $E$ can reroute that flow elsewhere; just whether $B$ can push flow backwards. Justify using 1-2 sentences.

\begin{proof}[Answer]
The amount of flow that $B$ can push back to $E$ is 2 because that is how much flow is being pushed along the edge $E \to B$.\\
\end{proof}



\vskip 10pt
\subsubsection{Problem 1\ref{c} (1 point)}
\item \label{c} Given the current flow configuration $f$, what is the maximum amount of flow that $D$ can push backwards to $E$? Do \textbf{not} consider whether $D$ can reroute that flow elsewhere; just whether $E$ can push flow backwards. Justify using 1-2 sentences.

\begin{proof}[Answer]
We can not push \textbf{any} flow from $D \to E$ since there is no flow going from $E \to D$.\\
\end{proof}


\vskip 10pt
\subsubsection{Problem 1\ref{d} (1 point)}
\item \label{d} How much \textit{additional} flow can be pushed along the flow-augmenting path $s \to B \to C \to E \to D \to t$? Do not include the current flow along these edges. Justify using 1-2 sentences.

\begin{proof}[Answer]
We can only push $2$ additional flow across this path. $s \to B$ we can push 5, $B \to C$ we can push 2, $C \to E$ we can push 7, $E \to D$ we can push 4, $D \to t$ we can push 6. We take the minimum ($2$) and use that as the maximum flow we can push across the path.\\
\end{proof}


\vskip 10pt
\subsubsection{Problem 1\ref{e} (1 point)}
\item \label{e} Find a second flow-augmenting path and indicate the maximum amount of additional flow that can be pushed along the path. Assume that the flow-augmenting path from part (d) has \textbf{not} been applied. Justify using 1-2 sentences.

\begin{proof}[Answer]
Another flow-augmenting path is the following: $s \to C \to E \to D \to t$.\\ This path has a maximum flow of $4$ because that is the minimum flow value of the edges $s \to C$ and $E \to D$.
\end{proof}
\end{enumerate}
\end{required}
\end{document} % NOTHING AFTER THIS LINE IS PART OF THE DOCUMENT
