\documentclass[11pt]{article} 
\usepackage[english]{babel}
\usepackage[utf8]{inputenc}
\usepackage[margin=0.5in]{geometry}
\usepackage{amsmath}
\usepackage{amsthm}
\usepackage{amsfonts}
\usepackage{amssymb}
\usepackage[usenames,dvipsnames]{xcolor}
\usepackage{graphicx}
\usepackage[siunitx]{circuitikz}
\usepackage{tikz}
\usepackage[colorinlistoftodos, color=orange!50]{todonotes}
\usepackage{hyperref}
\usepackage[numbers, square]{natbib}
\usepackage{fancybox}
\usepackage{epsfig}
\usepackage{soul}
\usepackage[framemethod=tikz]{mdframed}
\usepackage[shortlabels]{enumitem}
\usepackage[version=4]{mhchem}
\usepackage{multicol}

\usepackage{mathtools}
\usepackage{comment}
\usepackage{enumitem}
\usepackage[utf8]{inputenc}
\usepackage[linesnumbered,ruled,vlined]{algorithm2e}
\usepackage{listings}
\usepackage{color}
\usepackage[numbers]{natbib}
\usepackage{subfiles}
\usepackage{tkz-berge}


\newtheorem{prop}{Proposition}[section]
\newtheorem{thm}{Theorem}[section]
\newtheorem{lemma}{Lemma}[section]
\newtheorem{cor}{Corollary}[prop]

\theoremstyle{definition}
\newtheorem{definition}{Definition}

\theoremstyle{definition}
\newtheorem{required}{Problem}

\theoremstyle{definition}
\newtheorem{ex}{Example}


\setlength{\marginparwidth}{3.4cm}
%#########################################################

%To use symbols for footnotes
\renewcommand*{\thefootnote}{\fnsymbol{footnote}}
%To change footnotes back to numbers uncomment the following line
%\renewcommand*{\thefootnote}{\arabic{footnote}}

% Enable this command to adjust line spacing for inline math equations.
% \everymath{\displaystyle}

% _______ _____ _______ _      ______ 
%|__   __|_   _|__   __| |    |  ____|
%   | |    | |    | |  | |    | |__   
%   | |    | |    | |  | |    |  __|  
%   | |   _| |_   | |  | |____| |____ 
%   |_|  |_____|  |_|  |______|______|
%%%%%%%%%%%%%%%%%%%%%%%%%%%%%%%%%%%%%%%

\title{
\normalfont \normalsize 
\textsc{CSCI 3104 Spring 2023 \\
Instructors: Prof. Layer and Chandra Kanth Nagesh} \\
[10pt] 
\rule{\linewidth}{0.5pt} \\[6pt] 
\huge Midterm 2 Standard 25 - Dynamic Programming: Design a Dynamic Programming Algorithm
\rule{\linewidth}{2pt}  \\[10pt]
}
%\author{Your Name}
\date{}

\begin{document}

\maketitle


%%%%%%%%%%%%%%%%%%%%%%%%%
%%%%%%%%%%%%%%%%%%%%%%%%%%
%%%%%%%%%%FILL IN YOUR NAME%%%%%%%
%%%%%%%%%%AND STUDENT ID%%%%%%%%
%%%%%%%%%%%%%%%%%%%%%%%%%%
\noindent
Due Date \dotfill April 28th \\
Name \dotfill \textbf{Blake Raphael} \\
Student ID \dotfill \textbf{109752312} \\
Quiz Code (enter in Canvas to get access to the LaTeX template) \dotfill \textbf{ oXwHE } \\

\tableofcontents

\section{Instructions}
 \begin{itemize}
	\item The solutions \textbf{should be typed}, using proper mathematical notation. We cannot accept hand-written solutions. \href{http://ece.uprm.edu/~caceros/latex/introduction.pdf}{Here's a short intro to \LaTeX.}
	\item You should submit your work through the \textbf{class Canvas page} only. Please submit one PDF file, compiled using this \LaTeX \ template.
	\item You may not need a full page for your solutions; pagebreaks are there to help Gradescope automatically find where each problem is. Even if you do not attempt every problem, please submit this document with no fewer pages than the blank template (or Gradescope has issues with it).

	\item You \textbf{may not collaborate with other students}. \textbf{Copying from any source is an Honor Code violation. Furthermore, all submissions must be in your own words and reflect your understanding of the material.} If there is any confusion about this policy, it is your responsibility to clarify before the due date. 

	\item Posting to \textbf{any} service including, but not limited to Chegg, Discord, Reddit, StackExchange, etc., for help on an assignment is a violation of the Honor Code.

	\end{itemize}
\newpage

\section{Standard 25 - Dynamic Programming: Design a Dynamic Programming Algorithm}
\subsection{Problem \ref{prob}}\begin{required}\label{prob}

   
Consider the Min Path Sum problem that takes as input an $m \times n$ grid of non-negative integers and returns a path from the top left cell to the bottom right cell that minimizes the sum of the numbers in the path. At each step in the path, the only valid moves are down or right.

For example, the min path sum for the input $[[1,3,1],[1,5,1],[4,2,1]]$ is 7 for the path 
$1\rightarrow 3\rightarrow 1\rightarrow 1\rightarrow 1$, as shown in the table below.

\begin{center}
\begin{tabular}{|c|c|c|}
\hline
    {\bf 1} & {\bf 3} & {\bf 1} \\
\hline
    1 & 5 & {\bf 1} \\
\hline
    4 & 2 & {\bf 1} \\
\hline
\end{tabular}
\end{center}

Write down the mathematical recurrence, then work throuh and example for the the following input $[[1,2,3],[4,5,6]]$.


\end{required}
% Either type your answer in below, or uncomment the \includegraphics command
% and use it to insert an approprate image. Try experimenting with the scale
% 0.9 the width option to resize your image if necessary.

%\includegraphics[width=0.9\textwidth]{solution.jpg}

\begin{proof}[Answer]
Let $G$ be our grid of size $m \cdot n$, $G_{m,n}$ be the value of the grid space $(m , n)$ in grid $G$, and $F(m,n)$ be a step in the minimum sum path from the top left cell to the bottom right cell of grid $G$. As well as this, consider that cell $(1,1)$ is the top right corner of our grid $G$.\\

We get the following recurrence:\\

\begin{align*}
	F[m, n]= \begin{cases}
		\text{Empty} & : m = 0 \text{ or } n = 0 \\
		G_{m,n} &: m = 1 \text{ and } n = 1 \\
		min(F[m - 1, n], F[m , n -1], F[m - 1 , n - 1]) \to G_{m,n} &: m \geq 1, n \geq 1.\\
	\end{cases}
\end{align*}

Using our recurrence we get the following:\\

\begin{enumerate}
	\item  We start by adding $6$ to our list and recurring to the minimum of whatever $F[m - 1, n], F[m , n -1], F[m - 1 , n - 1]$ gives us. In this case we choose $F[m - 1 , n - 1]$.\\
	
	\item Now we have a list of $2 \to 6$ and recurse again to the minimum of whatever $F[m - 1, n], F[m , n -1], F[m - 1 , n - 1]$ gives us again. In this case we choose $F[m - 1, n]$ to get to our final cell.\\
	
	\item We now hit a base case of $m = 1$ and $n = 1$, so we just add the value of $G_{m, n}$ on to our list. We have our final list of $1 \to 2 \to 6$ which gives us a final minimum sum of 8.\\
\end{enumerate}
\end{proof}



%Include an Image: \includegraphics{ImageFileName}
%Include an Image and Rotate 90 degree: \includegraphics[angle=90]{ImageFileName}
%Include an Image, Rotate by 180 degrees, and scale by 50\% \includegraphics[scale=0.5, angle=90]{ImageFileName}


%%%%%%%%%%%%%%%%%%%%%%%%%%%%%%%%%%%%%%%%%%%%%%%%%%
\end{document} % NOTHING AFTER THIS LINE IS PART OF THE DOCUMENT
