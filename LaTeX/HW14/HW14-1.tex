\documentclass[11pt]{article}
\usepackage[english]{babel}
\usepackage[utf8]{inputenc}
\usepackage[margin=0.5in]{geometry}
\usepackage{amsmath}
\usepackage{amsthm}
\usepackage{amsfonts}
\usepackage{amssymb}
\usepackage[usenames,dvipsnames]{xcolor}
\usepackage{graphicx}
\usepackage[siunitx]{circuitikz}
\usepackage{tikz}
\usepackage[colorinlistoftodos, color=orange!50]{todonotes}
\usepackage{hyperref}
\usepackage[numbers, square]{natbib}
\usepackage{fancybox}
\usepackage{epsfig}
\usepackage{soul}
\usepackage[framemethod=tikz]{mdframed}
\usetikzlibrary{positioning, automata, backgrounds}
\usepackage[shortlabels]{enumitem}
\usepackage[version=4]{mhchem}
\usepackage{multicol}
\usepackage{forest}
\usepackage{mathtools}
\usepackage{comment}
\usepackage{enumitem}
\usepackage[utf8]{inputenc}
\usepackage[linesnumbered,ruled,vlined]{algorithm2e}
\usepackage{listings}
\usepackage{color}
\usepackage[numbers]{natbib}
\usepackage{subfiles}
\usepackage{tkz-berge}


\newcommand{\interval}[4]{\draw (#2, #1) -- (#3, #1); % Usage: \interval{height}{start}{end}{label}
\draw (#2, #1-0.11) -- (#2, #1+0.11); % draw left whisker
\draw (#3, #1-0.11) -- (#3, #1+0.11); % draw right whisker
\node[] at (#2-0.25, #1) {#4};
}

\newtheorem{prop}{Proposition}[section]
\newtheorem{thm}{Theorem}[section]
\newtheorem{lemma}{Lemma}[section]
\newtheorem{cor}{Corollary}[prop]

\theoremstyle{definition}
\newtheorem{definition}{Definition}

\theoremstyle{definition}
\newtheorem{required}{Problem}

\theoremstyle{definition}
\newtheorem{ex}{Example}


\setlength{\marginparwidth}{3.4cm}
%#########################################################

%To use symbols for footnotes
\renewcommand*{\thefootnote}{\fnsymbol{footnote}}
%To change footnotes back to numbers uncomment the following line
%\renewcommand*{\thefootnote}{\arabic{footnote}}

% Enable this command to adjust line spacing for inline math equations.
% \everymath{\displaystyle}

% _______ _____ _______ _      ______ 
%|__   __|_   _|__   __| |    |  ____|
%   | |    | |    | |  | |    | |__   
%   | |    | |    | |  | |    |  __|  
%   | |   _| |_   | |  | |____| |____ 
%   |_|  |_____|  |_|  |______|______|
%%%%%%%%%%%%%%%%%%%%%%%%%%%%%%%%%%%%%%%

\title{
\normalfont \normalsize 
\textsc{CSCI 3104 Spring 2023 \\ 
Instructors: Ryan Layer and Chandra Kanth Nagesh} \\
[10pt] 
\rule{\linewidth}{0.5pt} \\[6pt] 
\huge Homework 14 \\
\rule{\linewidth}{2pt}  \\[10pt]
}
\author{}
\date{}

\begin{document}

\definecolor {processblue}{cmyk}{0.96,0,0,0}
\definecolor{processred}{rgb}{200, 0, 0}
\definecolor{processgreen}{rgb}{0, 255, 0}
\DeclareGraphicsExtensions{.png}
\DeclareGraphicsExtensions{.gif}
\DeclareGraphicsExtensions{.jpg}

\maketitle


%%%%%%%%%%%%%%%%%%%%%%%%%
%%%%%%%%%%%%%%%%%%%%%%%%%%
%%%%%%%%%%FILL IN YOUR NAME%%%%%%%
%%%%%%%%%%AND STUDENT ID%%%%%%%%
%%%%%%%%%%%%%%%%%%%%%%%%%%
\noindent
Due Date \dotfill March 2, 2023 \\
Name \dotfill \textbf{Blake Raphael} \\
Student ID \dotfill \textbf{109752312} \\
Collaborators \dotfill \textbf{Alex Barry and Brody Cyphers}

\tableofcontents

\section{Instructions}
 \begin{itemize}
	\item The solutions \textbf{should be typed}, using proper mathematical notation. We cannot accept hand-written solutions. \href{http://ece.uprm.edu/~caceros/latex/introduction.pdf}{Here's a short intro to \LaTeX.}
	\item You should submit your work through the \textbf{class Gradescope page} only (linked from Canvas). Please submit one PDF file, compiled using this \LaTeX \ template.
	\item You may not need a full page for your solutions; pagebreaks are there to help Gradescope automatically find where each problem is. Even if you do not attempt every problem, please submit this document with no fewer pages than the blank template (or Gradescope has issues with it).

	\item You are welcome and encouraged to collaborate with your classmates, as well as consult outside resources. You must \textbf{cite your sources in this document.} \textbf{Copying from any source is an Honor Code violation. Furthermore, all submissions must be in your own words and reflect your understanding of the material.} If there is any confusion about this policy, it is your responsibility to clarify before the due date. 

	\item Posting to \textbf{any} service including, but not limited to Chegg, Reddit, StackExchange, etc., for help on an assignment is a violation of the Honor Code.

	\item You \textbf{must} virtually sign the Honor Code (see Section \ref{HonorCode}). Failure to do so will result in your assignment not being graded.
\end{itemize}


\section{Honor Code (Make Sure to Virtually Sign)} \label{HonorCode}

\begin{itemize}
\item My submission is in my own words and reflects my understanding of the material.
\item Any collaborations and external sources have been clearly cited in this document.
\item I have not posted to external services including, but not limited to Chegg, Reddit, StackExchange, etc.
\item I have neither copied nor provided others solutions they can copy.
\end{itemize}

%\noindent In the specified region below, clearly indicate that you have upheld the Honor Code. Then type your name. 

\begin{proof}[Agreed (I agree to above, Blake Raphael).]
%% Typing "I agree to the above," followed by your name is sufficient.
\end{proof}
\newpage
\section{Standard 14- Analyzing Code II: (Dependent nested loops)}
\subsection{Problem \ref{1} (4 points)}
\begin{required}
\label{1} Analyze the \textit{worst-case} runtime of the following algorithms. Clearly derive the runtime complexity function $T(n)$ for this algorithm, and then find a tight asymptotic bound for $T(n)$ (that is, find a function $f(n)$ such that $T(n) \in \Theta(f(n))$). Avoid heuristic arguments from 2270/2824 such as multiplying the complexities of nested loops.

\begin{verbatim}
	1: procedure foo_d(integer n):
	2:   for i = 1, i <= n
	3:       i = i + 2
	4:       print `outer'
	5:       for j = 1, j <= i
	6:           j = j + 3
	7:           print `inner'
\end{verbatim}

\begin{proof}	At line 2 we have a loop initialization with a single assignment of time complexity 1.\\ 
	The outer loop takes $n$ iterations, and at each iteration we have the following:\\
	\begin{itemize}[]\item Comparison $i \leq n$ which takes 1 step.
		\item Variable update which involves computation $i + 2$ and assignment $i = i + 2$, giving us a time cost of 2.
		\item We have a \begin{verbatim} print \end{verbatim} statement having a time cost of 1.
		\item Another loop initialization with a single assignment of time complexity 1.
		\item Finally we have another loop that takes $i$ iterations and at each iteration we have the following:\\
		\begin{itemize}[]\item Comparison $j \leq i$ which takes 1 step.
			\item Variable update which involves computation $j + 3$ and assignment $j = j + 3$, giving us a time cost of 2.
			\item Finally we have a \begin{verbatim} print \end{verbatim} statement having a time cost of 1.
		\end{itemize}
	\end{itemize}
	So the inner loop has time complexity 1 + $\displaystyle\sum_{j=1} ^{\frac{i - 1}{3}} (1 + 2 + 1)$ or $\displaystyle\sum_{j=1} ^{\frac{i - 1}{3}} 4j$\\
	The outer loop has time complexity $ 1 + \displaystyle\sum_{i=1} ^{\frac{n - 1}{2}} (1 + 2 + 1 + 1 + 4(\frac{i - 1}{3})) = 5 + \frac{4}{3}i - \frac{4}{3} = 5 + \frac{4(n^{2}+1)}{4} - \frac{4}{3} = 1 + \frac{5}{2}n - \frac{5}{2} + \frac{1}{6}n^{2}+ \frac{1}{6}-\frac{4}{3}$\\
	So we get $T(n) = 1 + \frac{5}{2}n - \frac{5}{2} + \frac{1}{6}n^{2}+ \frac{1}{6}-\frac{4}{3}$ thus, $T(n) \in \Theta(f(n^2))$.
\end{proof}

\end{required}
\end{document} % NOTHING AFTER THIS LINE IS PART OF THE DOCUMENT
