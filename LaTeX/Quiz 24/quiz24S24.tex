\documentclass[11pt]{article} 
\usepackage[english]{babel}
\usepackage[utf8]{inputenc}
\usepackage[margin=0.5in]{geometry}
\usepackage{amsmath}
\usepackage{amsthm}
\usepackage{amsfonts}
\usepackage{amssymb}
\usepackage[usenames,dvipsnames]{xcolor}
\usepackage{graphicx}
\usepackage[siunitx]{circuitikz}
\usepackage{tikz}
\usepackage[colorinlistoftodos, color=orange!50]{todonotes}
\usepackage{hyperref}
\usepackage[numbers, square]{natbib}
\usepackage{fancybox}
\usepackage{epsfig}
\usepackage{soul}
\usepackage[framemethod=tikz]{mdframed}
\usepackage[shortlabels]{enumitem}
\usepackage[version=4]{mhchem}
\usepackage{multicol}
\usepackage{algorithm}
\usepackage[noend]{algpseudocode}


\usepackage{mathtools}
\usepackage{comment}
\usepackage{enumitem}
\usepackage[utf8]{inputenc}
%\usepackage[linesnumbered,ruled,vlined]{algorithm2e}
\usepackage{listings}
\usepackage{color}
\usepackage[numbers]{natbib}
\usepackage{subfiles}
\usepackage{tkz-berge}


\newtheorem{prop}{Proposition}[section]
\newtheorem{thm}{Theorem}[section]
\newtheorem{lemma}{Lemma}[section]
\newtheorem{cor}{Corollary}[prop]

\theoremstyle{definition}
\newtheorem{definition}{Definition}

\theoremstyle{definition}
\newtheorem{required}{Problem}
\newtheorem*{requiredHC}{Problem HC}

\theoremstyle{definition}
\newtheorem{ex}{Example}


\setlength{\marginparwidth}{3.4cm}
%#########################################################

%To use symbols for footnotes
\renewcommand*{\thefootnote}{\fnsymbol{footnote}}
%To change footnotes back to numbers uncomment the following line
%\renewcommand*{\thefootnote}{\arabic{footnote}}

% Enable this command to adjust line spacing for inline math equations.
% \everymath{\displaystyle}

% _______ _____ _______ _      ______ 
%|__   __|_   _|__   __| |    |  ____|
%   | |    | |    | |  | |    | |__   
%   | |    | |    | |  | |    |  __|  
%   | |   _| |_   | |  | |____| |____ 
%   |_|  |_____|  |_|  |______|______|
%%%%%%%%%%%%%%%%%%%%%%%%%%%%%%%%%%%%%%%

\title{
\normalfont \normalsize 
\textsc{CSCI 3104 Spring 2023 \\ 
Instructors: Chandra Kanth Nagesh and Prof. Ryan Layer} \\
[10pt] 
\rule{\linewidth}{0.5pt} \\[6pt] 
\huge Quiz 24 Standard 24 --  DP -- Backtracking to find solutions \\
\rule{\linewidth}{2pt}  \\[10pt]
}
%\author{Your Name}
\date{}

\begin{document}
\definecolor {processblue}{cmyk}{0.96,0,0,0}
\definecolor{processred}{rgb}{200, 0, 0}
\definecolor{processgreen}{rgb}{0, 255, 0}
\DeclareGraphicsExtensions{.png}
\DeclareGraphicsExtensions{.gif}
\DeclareGraphicsExtensions{.jpg}

\maketitle


%%%%%%%%%%%%%%%%%%%%%%%%%
%%%%%%%%%%%%%%%%%%%%%%%%%%
%%%%%%%%%%FILL IN YOUR NAME%%%%%%%
%%%%%%%%%%AND STUDENT ID%%%%%%%%
%%%%%%%%%%%%%%%%%%%%%%%%%%
\noindent
Due Date \dotfill TODO \\
Name \dotfill \textbf{Blake Raphael} \\
Student ID \dotfill \textbf{109752312} \\
Quiz Code (enter in Canvas to get access to the LaTeX template) \dotfill \textbf{BSGAA}


\tableofcontents

\section*{Instructions}
\addcontentsline{toc}{section}{Instructions}
 \begin{itemize}
	\item You may either type your work using this template, or you may handwrite your work and embed it as an image in this template. \textbf{If you choose to handwrite your work, the image must be legible, and oriented so that we do not have to rotate our screens to grade your work.} We have included some helpful LaTeX commands for including and rotating images commented out near the end of the LaTeX template.
	\item You should submit your work through the \textbf{class Gradescope page} only. Please submit one PDF file, compiled using this \LaTeX \ template.
	\item You may not need a full page for your solutions; pagebreaks are there to help Gradescope automatically find where each problem is. Even if you do not attempt every problem, please submit this document with no fewer pages than the blank template (or Gradescope has issues with it).

	\item You \textbf{may not collaborate with other students}. \textbf{Copying from any source is an Honor Code violation. Furthermore, all submissions must be in your own words and reflect your understanding of the material.} If there is any confusion about this policy, it is your responsibility to clarify before the due date. 

	\item Posting to \textbf{any} service including, but not limited to Chegg, Discord, Reddit, StackExchange, etc., for help on an assignment is a violation of the Honor Code.

	\item You \textbf{must} virtually sign the Honor Code (see Section \ref{HonorCode}). Failure to do so will result in your assignment not being graded.
\end{itemize}


\newpage
\section*{Honor Code (Make Sure to Virtually Sign)} \label{HonorCode}
\addcontentsline{toc}{section}{Honor Code (Make Sure to Virtually Sign)}
\hypertarget{HonorCode}{}

\begin{requiredHC}
\begin{itemize}
\item My submission is in my own words and reflects my understanding of the material.
\item Any collaborations and external sources have been clearly cited in this document.
\item I have not posted to external services including, but not limited to Chegg, Reddit, StackExchange, etc.
\item I have neither copied nor provided others solutions they can copy.
\end{itemize}

%\noindent In the specified region below, clearly indicate that you have upheld the Honor Code. Then type your name. 
\end{requiredHC}

\begin{proof}[Agreed (I agree to the above, Blake Raphael).]
%% Typing "I agree to the above," followed by your name is sufficient.
\end{proof}


\newpage
\setcounter{section}{23}
\section{Standard 24: Dynamic Programming: Backtracking to find solutions}

\setcounter{required}{23}
\begin{required} 
Consider the \textsc{Longest Path in a DAG} problem on the following directed acyclic graph (the same as in HW8):

\begin{center}
	\begin {tikzpicture}[-latex ,auto ,node distance =2 cm and 3cm ,on grid ,
	semithick ,
	state/.style ={ circle ,top color =white , bottom color = processblue!20 ,
	draw,processblue , text=blue , minimum width =1 cm}]

	\node[state] (1) {$1$};
	\node[state] (2) [right = of 1] {$2$};
	\node[state] (3) [right = of 2] {$3$};
	\node[state] (4) [right = of 3] {$4$};
	\node[state] (5) [right = of 4] {$5$};
	\node[state] (6) [right = of 5] {$6$};
	
	\node[state] (7) [below = of 2] {7};
	\node[state] (8) [right = of 7] {8};	
	\node[state] (9) [right = of 8] {9};	
	\node[state] (10) [right = of 9] {10};	
	\node[state] (11) [right = of 10] {11};	
	
	\node[state] (12) [below = of 9] {12};
	\node[state] (13) [right = of 12] {13};
	\node[state] (14) [right = of 13] {14};
	
	\path (1) edge (2);
	\path (1) edge (7);
	\path (2) edge (3);
	\path (2) edge (7);
	\path (2) edge (8);
	\path (3) edge (4);
	\path (3) edge (8);
	\path (4) edge (5);
	\path (4) edge (9);
	\path (4) edge (10);
	\path (5) edge (6);
	\path (5) edge (11);
	\path (6) edge (11);
	\path (7) edge (8);
	\path (8) edge (9);
	\path (8) edge (12);
	\path (9) edge (10);
	\path (9) edge (12);
	\path (9) edge (13);
	\path (10) edge (11);
	\path (10) edge (13);
	\path (11) edge (14);
	\path (12) edge (13);
	\path (13) edge (14);
	\end{tikzpicture}  
\end{center}

The following is the dynamic programming table for the \emph{longest} path from each vertex $v$ to 14:

\begin{tabular}{|c|c|c|c|c|c|c|c|c|c|c|c|c|c|}
\hline
14 & 13 & 12 & 11 & 10 & 9 & 8 & 7 & 6 & 5 & 4 & 3 & 2 & 1 \\ \hline
0   & 1   & 2   & 1   & 2   & 3 & 4 & 5 & 2 & 3 & 4 & 5 & 6 & 7 \\ \hline
\end{tabular}

It was arrived at using this recurrence:
\[
L[j] = \begin{cases}
0 & j = 14 \\
1 + \max\{L[k] : (j,k) \in E(G)\} & j < 14
\end{cases}.
\]

From the table, \textbf{clearly indicate the steps you take to backtrace to find the longest path $1 \to 14$.} Be sure to indicate what choices you are considering at each step and which choice was chosen. 
\end{required}

\begin{proof}[Answer]
Here is the path taken:\\
\begin{enumerate}
	\item We start by considering 2 or 7 from node 1. We choose node 2.\\
	\item We then consider nodes 3,7, or 8. We choose node 7.\\
	\item We then consider node 8 and choose node 8.\\
	\item We then consider nodes 9 and 12. We choose node 9.\\
	\item We then consider nodes 10, 12, and 13. We choose node 10.\\
	\item We now consider nodes 11 and 13. We choose node 11.\\
	\item Finally we consider node 14 and choose node 14.\\
\end{enumerate}
This gives us the following path: $1 \to 2 \to 7 \to 8 \to 9 \to 10 \to 11 \to 14$ which, by using our recurrence relation, gives us a path length of 7 which is our longest path from $1 \to 14$.\\

\end{proof}

%%%%%%%%%%%%%%%%%%%%%%%%%%%%%%%%%%%%%%%%%%%%%%%%%%
\end{document} % NOTHING AFTER THIS LINE IS PART OF THE DOCUMENT



