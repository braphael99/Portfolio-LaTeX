\documentclass[11pt]{article}
\usepackage[english]{babel}
\usepackage[utf8]{inputenc}
\usepackage[margin=0.5in]{geometry}
\usepackage{amsmath}
\usepackage{amsthm}
\usepackage{amsfonts}
\usepackage{amssymb}
\usepackage[usenames,dvipsnames]{xcolor}
\usepackage{graphicx}
\usepackage[siunitx]{circuitikz}
\usepackage{tikz}
\usepackage[colorinlistoftodos, color=orange!50]{todonotes}
\usepackage{hyperref}
\usepackage[numbers, square]{natbib}
\usepackage{fancybox}
\usepackage{epsfig}
\usepackage{soul}
\usepackage[framemethod=tikz]{mdframed}
\usetikzlibrary{positioning, automata, backgrounds}
\usepackage[shortlabels]{enumitem}
\usepackage[version=4]{mhchem}
\usepackage{multicol}
\usepackage{forest}
\usepackage{mathtools}
\usepackage{comment}
\usepackage{enumitem}
\usepackage[utf8]{inputenc}
%\usepackage[linesnumbered,ruled,vlined]{algorithm2e}
\usepackage{algorithm}
\usepackage[noend]{algpseudocode}
\usepackage{listings}
\usepackage{color}
\usepackage[numbers]{natbib}
\usepackage{subfiles}
\usepackage{tkz-berge}


\newcommand{\interval}[4]{\draw (#2, #1) -- (#3, #1); % Usage: \interval{height}{start}{end}{label}
\draw (#2, #1-0.11) -- (#2, #1+0.11); % draw left whisker
\draw (#3, #1-0.11) -- (#3, #1+0.11); % draw right whisker
\node[] at (#2-0.25, #1) {#4};
}

\newtheorem{prop}{Proposition}[section]
\newtheorem{thm}{Theorem}[section]
\newtheorem{lemma}{Lemma}[section]
\newtheorem{cor}{Corollary}[prop]

\theoremstyle{definition}
\newtheorem{definition}{Definition}

\theoremstyle{definition}
\newtheorem{required}{Problem}

\theoremstyle{definition}
\newtheorem{ex}{Example}


\setlength{\marginparwidth}{3.4cm}
%#########################################################

%To use symbols for footnotes
\renewcommand*{\thefootnote}{\fnsymbol{footnote}}
%To change footnotes back to numbers uncomment the following line
%\renewcommand*{\thefootnote}{\arabic{footnote}}

% Enable this command to adjust line spacing for inline math equations.
% \everymath{\displaystyle}

% _______ _____ _______ _      ______ 
%|__   __|_   _|__   __| |    |  ____|
%   | |    | |    | |  | |    | |__   
%   | |    | |    | |  | |    |  __|  
%   | |   _| |_   | |  | |____| |____ 
%   |_|  |_____|  |_|  |______|______|
%%%%%%%%%%%%%%%%%%%%%%%%%%%%%%%%%%%%%%%

\title{
\normalfont \normalsize 
\textsc{CSCI 3104 Spring 2023 \\ 
Instructors: Ryan Layer and Chandra Kanth Nagesh} \\
[10pt] 
\rule{\linewidth}{0.5pt} \\[6pt] 
\huge Homework 22 \\
\rule{\linewidth}{2pt}  \\[10pt]
}
\author{}
\date{}

\begin{document}

\definecolor {processblue}{cmyk}{0.96,0,0,0}
\definecolor{processred}{rgb}{200, 0, 0}
\definecolor{processgreen}{rgb}{0, 255, 0}
\DeclareGraphicsExtensions{.png}
\DeclareGraphicsExtensions{.gif}
\DeclareGraphicsExtensions{.jpg}

\maketitle


%%%%%%%%%%%%%%%%%%%%%%%%%
%%%%%%%%%%%%%%%%%%%%%%%%%%
%%%%%%%%%%FILL IN YOUR NAME%%%%%%%
%%%%%%%%%%AND STUDENT ID%%%%%%%%
%%%%%%%%%%%%%%%%%%%%%%%%%%
\noindent
Due Date \dotfill Apring 6, 2023 \\
Name \dotfill \textbf{Blake Raphael} \\
Student ID \dotfill \textbf{109752312} \\
Collaborators \dotfill \textbf{Alex Barry, Brody Cyphers, and Ben Kohav}

\tableofcontents

\section{Instructions}
 \begin{itemize}
	\item The solutions \textbf{should be typed}, using proper mathematical notation. We cannot accept hand-written solutions. \href{http://ece.uprm.edu/~caceros/latex/introduction.pdf}{Here's a short intro to \LaTeX.}
	\item You should submit your work through the \textbf{class Gradescope page} only (linked from Canvas). Please submit one PDF file, compiled using this \LaTeX \ template.
	\item You may not need a full page for your solutions; pagebreaks are there to help Gradescope automatically find where each problem is. Even if you do not attempt every problem, please submit this document with no fewer pages than the blank template (or Gradescope has issues with it).

	\item You are welcome and encouraged to collaborate with your classmates, as well as consult outside resources. You must \textbf{cite your sources in this document.} \textbf{Copying from any source is an Honor Code violation. Furthermore, all submissions must be in your own words and reflect your understanding of the material.} If there is any confusion about this policy, it is your responsibility to clarify before the due date. 

	\item Posting to \textbf{any} service including, but not limited to Chegg, Reddit, StackExchange, etc., for help on an assignment is a violation of the Honor Code.

	\item You \textbf{must} virtually sign the Honor Code (see Section \ref{HonorCode}). Failure to do so will result in your assignment not being graded.
\end{itemize}


\section{Honor Code (Make Sure to Virtually Sign)} \label{HonorCode}

\begin{itemize}
\item My submission is in my own words and reflects my understanding of the material.
\item Any collaborations and external sources have been clearly cited in this document.
\item I have not posted to external services including, but not limited to Chegg, Reddit, StackExchange, etc.
\item I have neither copied nor provided others solutions they can copy.
\end{itemize}

%\noindent In the specified region below, clearly indicate that you have upheld the Honor Code. Then type your name. 

\begin{proof}[Agreed (I agree to the above, Blake Raphael).]
%% Typing "I agree to the above," followed by your name is sufficient.
\end{proof}
\newpage
\section{Standard 22 -- Dynamic Programming: Write Down Recurrences}

\subsection{Problem \ref{DP1} (2 Points)}
\begin{required} \label{DP1}
Suppose we have an $m$-letter alphabet $\Sigma = \{0, 1, \ldots, m-1\}$. Let $W_{n}$ be the set of strings $\omega \in \Sigma^{n}$ such that $\omega$ does not have $00$ as a substring. Let $f_{n} := |W_{n}|$. Write down an \textbf{explicit recurrence for $f_{n}$, including the base cases.} Clearly justify each recursive term.
\end{required}
    
\begin{proof}[Answer]
%%%%%%%%%%%%%%%%%%%%%%%%%%%%%%%%%%%%%%%%%%%%%%%%%%
% YOUR ANSWER GOES HERE                          %
%%%%%%%%%%%%%%%%%%%%%%%%%%%%%%%%%%%%%%%%%%%%%%%%%%
\begin{align*}
	f_n = \begin{cases}
		1 & : n = 0, \\
		m & : n = 1, \\
		(m - 1) f_{n - 1} + (m - 1)f_{n - 2} &: \text{else}.\\
	\end{cases}
\end{align*}

For base case of 1, when we have $n = 0$, there is only one string to consider, and it is not 00, so we have size 1. When $n = 1$, we only have one instance of $\Sigma = \{0, 1, \ldots, m-1\}$ to consider, which does not have 00, so we have $m$. Finally, when we have $n > 1$, we have $n$ of $\Sigma = \{0, 1, \ldots, m-1\}$ to consider, so we subtract the 00 out of however many strings we are considering, so we get $(m - 1) f_{n - 1} + (m - 1)f_{n - 2}$.\\
\end{proof}

\newpage
\subsection{Problem \ref{DP2} (2 Points)}
\begin{required} \label{DP2}

Suppose we have the alphabet $\Sigma = \{x, y\}$. For $n \geq 0$, let $W_{n}$ be the set of strings $\omega \in \{x, y\}^{n}$ where $\omega$ contains $yyy$ as a substring. Let $f_{n} := |W_{n}|$. Write down an explicit recurrence for $f_{n}$, including the base cases. Clearly justify each recursive term.
\end{required}

\begin{proof}[Answer]
When considering the alphabet $\Sigma = \{x, y\}$ and how many strings contain the subset $yyy$, we have one base case to account for and that is when $n \leq 2$. This is because it is impossible to have the string $yyy$ with a length of 2. \\

Now we consider our recursive cases:\\
\textbf{Case 1:} Where the $n$th character of $\omega_n$ is $x$.\\

We have a string $\omega$ which is $(\omega_1), (\omega_2), ..., (x)\text{ or $\omega_n$}$.\\ 
If $x$ is the last letter of our string, then we only have to consider sub-strings to the left of our final letter, or $f_{n-1}$.\\

\textbf{Case 2:} Where the $n$th and $n-1$th character of $\omega_n$ are $xy$.\\

We have a string $\omega$ which is $(\omega_1), (\omega_2), ...(x)\text{ or $\omega_{n-1}$}, (y)\text{ or $\omega_n$}$.\\ 
If $xy$ are the last letters of our string, then we only have to consider sub-strings to the left of our final letters, or $f_{n-2}$.\\

\textbf{Case 3:} Where the $n$th, $n-1$th and $n - 2$th characters of $\omega_n$ are $xyy$.\\

We have a string $\omega$ which is $(\omega_1), (\omega_2), ...(x)\text{ or $\omega_{n-2}$}, (y)\text{ or $\omega_{n-1}$}, (y)\text{ or $\omega_n$}$.\\ 
If $xyy$ are the last letters of our string, then we only have to consider sub-strings to the left of our final letters, or $f_{n-3}$.\\

\textbf{Case 4:} Where the $n$th, $n-1$th and $n - 2$th characters of $\omega_n$ are $yyy$.\\

We have a string $\omega$ which is $(\omega_1), (\omega_2), ...(y)\text{ or $\omega_{n-2}$}, (y)\text{ or $\omega_{n-1}$}, (y)\text{ or $\omega_n$}$.\\ 
If $yyy$ are the last letters of our string, then we have found $yyy$. Next, we have to consider sub-strings to the left of our final letter to see if there are any combinations that creat $yyy$. This is equal to $2^{n-3}$ because we are considering all combinations of $\omega$ from $n-3$ and only have 2 letters to choose from.\\

Altogether, we have the following:\\

\begin{align*}
	f_n = \begin{cases}
		0 & : n \leq 2, \\
		f_{n-1} + f_{n-2} + f_{n - 3} + 2^{n - 3} &: \text{else}.\\
	\end{cases}
\end{align*}
%%%%%%%%%%%%%%%%%%%%%%%%%%%%%%%%%%%%%%%%%%%%%%%%%%
% YOUR ANSWER GOES HERE                          %
%%%%%%%%%%%%%%%%%%%%%%%%%%%%%%%%%%%%%%%%%%%%%%%%%%
\end{proof}
\end{document} % NOTHING AFTER THIS LINE IS PART OF THE DOCUMENT
