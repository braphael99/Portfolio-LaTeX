\documentclass[11pt]{article}
\usepackage[english]{babel}
\usepackage[utf8]{inputenc}
\usepackage[margin=0.5in]{geometry}
\usepackage{amsmath}
\usepackage{amsthm}
\usepackage{amsfonts}
\usepackage{amssymb}
\usepackage[usenames,dvipsnames]{xcolor}
\usepackage{graphicx}
\usepackage[siunitx]{circuitikz}
\usepackage{tikz}
\usepackage[colorinlistoftodos, color=orange!50]{todonotes}
\usepackage{hyperref}
\usepackage[numbers, square]{natbib}
\usepackage{fancybox}
\usepackage{epsfig}
\usepackage{soul}
\usepackage[framemethod=tikz]{mdframed}
\usetikzlibrary{positioning, automata, backgrounds}
\usepackage[shortlabels]{enumitem}
\usepackage[version=4]{mhchem}
\usepackage{multicol}
\usepackage{forest}
\usepackage{mathtools}
\usepackage{comment}
\usepackage{enumitem}
\usepackage[utf8]{inputenc}
%\usepackage[linesnumbered,ruled,vlined]{algorithm2e}
\usepackage{algorithm}
\usepackage[noend]{algpseudocode}
\usepackage{listings}
\usepackage{color}
\usepackage[numbers]{natbib}
\usepackage{subfiles}
\usepackage{tkz-berge}


\newcommand{\interval}[4]{\draw (#2, #1) -- (#3, #1); % Usage: \interval{height}{start}{end}{label}
\draw (#2, #1-0.11) -- (#2, #1+0.11); % draw left whisker
\draw (#3, #1-0.11) -- (#3, #1+0.11); % draw right whisker
\node[] at (#2-0.25, #1) {#4};
}

\newtheorem{prop}{Proposition}[section]
\newtheorem{thm}{Theorem}[section]
\newtheorem{lemma}{Lemma}[section]
\newtheorem{cor}{Corollary}[prop]

\theoremstyle{definition}
\newtheorem{definition}{Definition}

\theoremstyle{definition}
\newtheorem{required}{Problem}

\theoremstyle{definition}
\newtheorem{ex}{Example}


\setlength{\marginparwidth}{3.4cm}
%#########################################################

%To use symbols for footnotes
\renewcommand*{\thefootnote}{\fnsymbol{footnote}}
%To change footnotes back to numbers uncomment the following line
%\renewcommand*{\thefootnote}{\arabic{footnote}}

% Enable this command to adjust line spacing for inline math equations.
% \everymath{\displaystyle}

% _______ _____ _______ _      ______ 
%|__   __|_   _|__   __| |    |  ____|
%   | |    | |    | |  | |    | |__   
%   | |    | |    | |  | |    |  __|  
%   | |   _| |_   | |  | |____| |____ 
%   |_|  |_____|  |_|  |______|______|
%%%%%%%%%%%%%%%%%%%%%%%%%%%%%%%%%%%%%%%

\title{
\normalfont \normalsize 
\textsc{CSCI 3104 Spring 2023 \\ 
Instructors: Ryan Layer and Chandra Kanth Nagesh} \\
[10pt] 
\rule{\linewidth}{0.5pt} \\[6pt] 
\huge Homework 20 \\
\rule{\linewidth}{2pt}  \\[10pt]
}
\author{}
\date{}

\begin{document}

\definecolor {processblue}{cmyk}{0.96,0,0,0}
\definecolor{processred}{rgb}{200, 0, 0}
\definecolor{processgreen}{rgb}{0, 255, 0}
\DeclareGraphicsExtensions{.png}
\DeclareGraphicsExtensions{.gif}
\DeclareGraphicsExtensions{.jpg}

\maketitle


%%%%%%%%%%%%%%%%%%%%%%%%%
%%%%%%%%%%%%%%%%%%%%%%%%%%
%%%%%%%%%%FILL IN YOUR NAME%%%%%%%
%%%%%%%%%%AND STUDENT ID%%%%%%%%
%%%%%%%%%%%%%%%%%%%%%%%%%%
\noindent
Due Date \dotfill March 23, 2023 \\
Name \dotfill \textbf{Blake Raphael} \\
Student ID \dotfill \textbf{109752312} \\
Collaborators \dotfill \textbf{Alex Barry, Brody Cyphers, and Ben Kohav}

\tableofcontents

\section{Instructions}
 \begin{itemize}
	\item The solutions \textbf{should be typed}, using proper mathematical notation. We cannot accept hand-written solutions. \href{http://ece.uprm.edu/~caceros/latex/introduction.pdf}{Here's a short intro to \LaTeX.}
	\item You should submit your work through the \textbf{class Gradescope page} only (linked from Canvas). Please submit one PDF file, compiled using this \LaTeX \ template.
	\item You may not need a full page for your solutions; pagebreaks are there to help Gradescope automatically find where each problem is. Even if you do not attempt every problem, please submit this document with no fewer pages than the blank template (or Gradescope has issues with it).

	\item You are welcome and encouraged to collaborate with your classmates, as well as consult outside resources. You must \textbf{cite your sources in this document.} \textbf{Copying from any source is an Honor Code violation. Furthermore, all submissions must be in your own words and reflect your understanding of the material.} If there is any confusion about this policy, it is your responsibility to clarify before the due date. 

	\item Posting to \textbf{any} service including, but not limited to Chegg, Reddit, StackExchange, etc., for help on an assignment is a violation of the Honor Code.

	\item You \textbf{must} virtually sign the Honor Code (see Section \ref{HonorCode}). Failure to do so will result in your assignment not being graded.
\end{itemize}


\section{Honor Code (Make Sure to Virtually Sign)} \label{HonorCode}

\begin{itemize}
\item My submission is in my own words and reflects my understanding of the material.
\item Any collaborations and external sources have been clearly cited in this document.
\item I have not posted to external services including, but not limited to Chegg, Reddit, StackExchange, etc.
\item I have neither copied nor provided others solutions they can copy.
\end{itemize}

%\noindent In the specified region below, clearly indicate that you have upheld the Honor Code. Then type your name. 

\begin{proof}[Agreed (I agree to the above, Blake Raphael).]
%% Typing "I agree to the above," followed by your name is sufficient.
\end{proof}
\newpage
\section{Standard 20: Quicksort}
\subsection{Problem \ref{qbest} (0.5 Points)}
\begin{required}\label{qbest}
Write down a recurrence relation that models the {\bf best case} running time of Quicksort, i.e. the case where {\sc Partition} selects the {\bf median} element at each iteration.

\begin{proof}[Answer]
% YOUR ANSWER BELOW
\begin{align*}
T(n) = \begin{cases}
\Theta(1) & : n \leq 1, \\
2T(n/2) + \Theta(n) & : n > 1.
\end{cases}
\end{align*}
\end{proof}
\end{required}

\newpage 
\subsection{Problem \ref{qworst} (0.5 Points)}
\begin{required}\label{qworst}
Write down a recurrence relation that models the {\bf worst case} running time of Quicksort, i.e. the case where {\sc Partition} selects the {\bf last} element at each iteration.

\begin{proof}[Answer]
% YOUR ANSWER BELOW
\begin{align*}
T(n) = \begin{cases}
\Theta(1) & : n \leq 1, \\
T(1) + T(n-1) + \Theta(n) & : n > 1.
\end{cases}
\end{align*}
\end{proof}
\end{required}

\newpage
\subsection{Problem \ref{qrec} (3 Points) }
\begin{required}\label{qrec}
Suppose that we modify {\sc Partition}($A,s,e$) so that it chooses the median element of $A[s..e]$ in calls that occur in nodes of even depth of the recursion tree of a call {\sc Quicksort}($A[1, \ldots, n],1, n$), and it chooses the maximum element of $A[s..e]$ in calls that occur in nodes of odd depth of this recursion tree. \\
  
\noindent Assume that the running time of this modified {\sc Partition} is still $\Theta(n)$ on any subarray of length $n$. You may assume that the root of a recursion tree starts at level $0$ (which is an even number), its children are at level 1, etc. \\
  
\noindent \textbf{Your job} is to write down a recurrence relation for the running time of this version of {\sc Quicksort} given an array $n$ distinct elements and solve it asymptotically, i.e.\ give your answer as $\Theta(f(n))$ for some function $f(n)$. Show your work.

\begin{proof}[Answer]
Here is the recurrence relation for this version of quicksort:\\

\begin{align*}
	T(n) = \begin{cases}
		\Theta(1) & : n \leq 1, \\
		T(1) + T(n-1) + \Theta(n) & : \text{If } n \text{ is odd}. \\
		2T(n/2) + \Theta(n) & : \text{If } n \text{ is even}.
	\end{cases}
\end{align*}

We get from the earlier problems that for evens, $k = \log_2(n)$ and $T(n) = \displaystyle\sum_{i=0} ^{\log_2(n)} (n) = n\log(n)$\\
So $T(n) \in \Theta(n\log(n))$

We also get from earlier problems that the runtime for odd complexity is the following: \\

\begin{align*}
	T(n) &= T(n-2) + \Theta(n-1) + \Theta(n))\\
	&= T(n - 3) + \Theta(n - 2) + \Theta(n - 1) + \Theta(n)\\
	&= T(n - k) + \Theta(n - (k + 1)) + ... + \Theta(n)
\end{align*}\\

So $T(n) \in \Theta(n^2)$\\

We use evens as the upper bound, so $T(n) \in \Theta(n\log(n))$\\


\end{proof}

\end{required}

\end{document} % NOTHING AFTER THIS LINE IS PART OF THE DOCUMENT
