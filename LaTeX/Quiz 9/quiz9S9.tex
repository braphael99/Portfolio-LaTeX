\documentclass[11pt]{article} 
\usepackage[english]{babel}
\usepackage[utf8]{inputenc}
\usepackage[margin=0.5in]{geometry}
\usepackage{amsmath}
\usepackage{amsthm}
\usepackage{amsfonts}
\usepackage{amssymb}
\usepackage[usenames,dvipsnames]{xcolor}
\usepackage{graphicx}
\usepackage[siunitx]{circuitikz}
\usepackage{tikz}
\usepackage[colorinlistoftodos, color=orange!50]{todonotes}
\usepackage{hyperref}
\usepackage[numbers, square]{natbib}
\usepackage{fancybox}
\usepackage{epsfig}
\usepackage{soul}
\usepackage[framemethod=tikz]{mdframed}
\usepackage[shortlabels]{enumitem}
\usepackage[version=4]{mhchem}
\usepackage{multicol}

\usepackage{mathtools}
\usepackage{comment}
\usepackage{enumitem}
\usepackage[utf8]{inputenc}
\usepackage[linesnumbered,ruled,vlined]{algorithm2e}
\usepackage{listings}
\usepackage{color}
\usepackage[numbers]{natbib}
\usepackage{subfiles}
\usepackage{tkz-berge}


\newtheorem{prop}{Proposition}[section]
\newtheorem{thm}{Theorem}[section]
\newtheorem{lemma}{Lemma}[section]
\newtheorem{cor}{Corollary}[prop]

\theoremstyle{definition}
\newtheorem{definition}{Definition}

\theoremstyle{definition}
\newtheorem{required}{Problem}
\newtheorem*{requiredHC}{Problem HC}

\theoremstyle{definition}
\newtheorem{ex}{Example}


\setlength{\marginparwidth}{3.4cm}
%#########################################################

%To use symbols for footnotes
\renewcommand*{\thefootnote}{\fnsymbol{footnote}}
%To change footnotes back to numbers uncomment the following line
%\renewcommand*{\thefootnote}{\arabic{footnote}}

% Enable this command to adjust line spacing for inline math equations.
% \everymath{\displaystyle}

% _______ _____ _______ _      ______ 
%|__   __|_   _|__   __| |    |  ____|
%   | |    | |    | |  | |    | |__   
%   | |    | |    | |  | |    |  __|  
%   | |   _| |_   | |  | |____| |____ 
%   |_|  |_____|  |_|  |______|______|
%%%%%%%%%%%%%%%%%%%%%%%%%%%%%%%%%%%%%%%

\title{
\normalfont \normalsize 
\textsc{CSCI 3104 Spring 2023 \\ 
Instructors: Chandra Kanth Nagesh and Prof. Ryan Layer} \\
[10pt] 
\rule{\linewidth}{0.5pt} \\[6pt] 
\huge Quiz 9 Standard 9 - Network Flow Terminology \\
\rule{\linewidth}{2pt}  \\[10pt]
}
%\author{Your Name}
\date{}

\begin{document}
\definecolor {processblue}{cmyk}{0.96,0,0,0}
\definecolor{processred}{rgb}{200, 0, 0}
\definecolor{processgreen}{rgb}{0, 255, 0}
\DeclareGraphicsExtensions{.png}
\DeclareGraphicsExtensions{.gif}
\DeclareGraphicsExtensions{.jpg}

\maketitle


%%%%%%%%%%%%%%%%%%%%%%%%%
%%%%%%%%%%%%%%%%%%%%%%%%%%
%%%%%%%%%%FILL IN YOUR NAME%%%%%%%
%%%%%%%%%%AND STUDENT ID%%%%%%%%
%%%%%%%%%%%%%%%%%%%%%%%%%%
\noindent
Due Date \dotfill TODO \\
Name \dotfill \textbf{Blake Raphael} \\
Student ID \dotfill \textbf{109752312} \\
Quiz Code (enter in Canvas to get access to the LaTeX template) \dotfill \textbf{TYHHH}


\tableofcontents

\section*{Instructions}
\addcontentsline{toc}{section}{Instructions}
 \begin{itemize}
	\item You may either type your work using this template, or you may handwrite your work and embed it as an image in this template. \textbf{If you choose to handwrite your work, the image must be legible, and oriented so that we do not have to rotate our screens to grade your work.} We have included some helpful LaTeX commands for including and rotating images commented out near the end of the LaTeX template.
	\item You should submit your work through the \textbf{class Gradescope page} only. Please submit one PDF file, compiled using this \LaTeX \ template.
	\item You may not need a full page for your solutions; pagebreaks are there to help Gradescope automatically find where each problem is. Even if you do not attempt every problem, please submit this document with no fewer pages than the blank template (or Gradescope has issues with it).

	\item You \textbf{may not collaborate with other students}. \textbf{Copying from any source is an Honor Code violation. Furthermore, all submissions must be in your own words and reflect your understanding of the material.} If there is any confusion about this policy, it is your responsibility to clarify before the due date. 

	\item Posting to \textbf{any} service including, but not limited to Chegg, Discord, Reddit, StackExchange, etc., for help on an assignment is a violation of the Honor Code.

	\item You \textbf{must} virtually sign the Honor Code (see Section \ref{HonorCode}). Failure to do so will result in your assignment not being graded.
\end{itemize}


\newpage
\section*{Honor Code (Make Sure to Virtually Sign)} \label{HonorCode}
\addcontentsline{toc}{section}{Honor Code (Make Sure to Virtually Sign)}
\hypertarget{HonorCode}{}

\begin{requiredHC}
\begin{itemize}
\item My submission is in my own words and reflects my understanding of the material.
\item Any collaborations and external sources have been clearly cited in this document.
\item I have not posted to external services including, but not limited to Chegg, Reddit, StackExchange, etc.
\item I have neither copied nor provided others solutions they can copy.
\end{itemize}

%\noindent In the specified region below, clearly indicate that you have upheld the Honor Code. Then type your name. 
\end{requiredHC}

\begin{proof}[Agreed (I agree to the above, Blake Raphael).]
%% Typing "I agree to the above," followed by your name is sufficient.
\end{proof}


\newpage
\setcounter{section}{8}
\section{Standard 9: Network flow terminology (4 points)}

\setcounter{required}{8}
\begin{required} 
Consider the following flow network, with the following flow configuration $f$ as indicated below. 
\begin{center}
	\begin {tikzpicture}[-latex ,auto ,node distance =2 cm and 3cm ,on grid ,
	semithick ,
	state/.style ={ circle ,top color =white , bottom color = processblue!40 ,
	draw,processblue , text=blue , minimum width =1 cm}]

	\node[state] (A) {$s$};
	\node[state] (B) [above right = of A] {$B$};
	\node[state] (C) [below right = of A] {$C$};
	\node[state] (D) [right = of B] {$D$};
	\node[state] (E) [right = of C] {$E$};
	\node[state] (t) [below right = of D] {$t$};
	
	\path (A) edge node[below] {$3 / 5$} (B);
	\path (A) edge node[above] {$2 / 6$} (C);
	\path (B) edge node[left] {$1 / 1$} (C);
	\path (B) edge node[above] {$3 / 4$} (D);
	\path (E) edge node[right] {$1 /  5$} (B);
	\path (C) edge node[above] {$3 / 7$} (E);
	\path (D) edge node[right] {$3 / 3$} (E);
	\path (D) edge node[below] {$0 / 3$} (t);
	\path (E) edge node[above] {$4 / 6$} (t);
	
	\end{tikzpicture}  
\end{center}

There are four parts to this question, (a)--(d); be sure to answer all four!

\end{required}

\renewcommand{\theenumi}{\alph{enumi}}
\begin{enumerate}
\item There is one vertex at which the above configuration is \emph{not} a flow. Specify the vertex at which it is not a flow, and argue why it is not a flow there---say what you are calculating and show your calculation.

\begin{proof}[Answer]
There is not a flow at vertex $E$. This is because the amount of flow entering the vertex and leaving the vertex are not equal. From vertex $C$, $E$ is receiving $3$. From vertex $D$, $E$ is receiving $3$. Out of vertex $E$, we have $1$ flow going to $B$ and $4$ flow going to $t$. $3 + 3 = 6$ and $4 + 1 = 5$, but $6 \ne 5$ so we do not have a flow.
\end{proof}

\vfill
\item Consider the path $s \to B \to D \to t$. Is it an augmenting path? If so, specify how much additional flow can be pushed along this path, and show your calculation. If not, explain why not.

\begin{proof}[Answer]
$s \to B \to D \to t$ is an augmenting path. We can push $1$ more flow through it. This is because we take the minimum of the remaining capacity of the edges connecting the nodes in the path. We can push $2$ more through $s \to B$, $1$ more through $B \to D$, and $3$ more through $D \to t$. The minimum is $1$, so we can only push 1 more flow through.
\end{proof}

\vfill
\newpage 

\item Consider the path $s \to B \to E \to t$. Is it an augmenting path? If so, specify how much additional flow can be pushed along this path, and show your calculation. If not, explain why not.

\begin{proof}[Answer]
$s \to B \to E \to t$ is \textbf{NOT} an augmenting path because $B \to E$ is not the correct direction for flow. The correct flow edge is actually $E \to B$, therefore we cannot push any flow, let alone more, through the specified path.
\end{proof}

\vfill
\item Consider the cut $\{(s, B)\}$ (in terms of sets of vertices, this is the cut that divides $\{s, B\}$ from $\{C,D,E,t\}$). What is the capacity of this cut? Explain your answer / show your calculation.

\begin{proof}[Answer]
The cut $\{(s, B)\}$ has capacity of $5$. We calculate this by taking the sum of the capacities of the edges within our cut. Since our cut only has one edge, the capacity of that edge would be the capacity of our cut, so we get $5$
\end{proof}
\vfill

\end{enumerate}


%%%%%%%%%%%%%%%%%%%%%%%%%%%%%%%%%%%%%%%%%%%%%%%%%%
\end{document} % NOTHING AFTER THIS LINE IS PART OF THE DOCUMENT



