\documentclass[11pt]{article}
\usepackage[english]{babel}
\usepackage[utf8]{inputenc}
\usepackage[margin=0.5in]{geometry}
\usepackage{amsmath}
\usepackage{amsthm}
\usepackage{amsfonts}
\usepackage{amssymb}
\usepackage[usenames,dvipsnames]{xcolor}
\usepackage{graphicx}
\usepackage[siunitx]{circuitikz}
\usepackage{tikz}
\usepackage[colorinlistoftodos, color=orange!50]{todonotes}
\usepackage{hyperref}
\usepackage[numbers, square]{natbib}
\usepackage{fancybox}
\usepackage{epsfig}
\usepackage{soul}
\usepackage[framemethod=tikz]{mdframed}
\usetikzlibrary{positioning, automata, backgrounds}
\usepackage[shortlabels]{enumitem}
\usepackage[version=4]{mhchem}
\usepackage{multicol}
\usepackage{forest}
\usepackage{mathtools}
\usepackage{comment}
\usepackage{enumitem}
\usepackage[utf8]{inputenc}
\usepackage[linesnumbered,ruled,vlined]{algorithm2e}
\usepackage{listings}
\usepackage{color}
\usepackage[numbers]{natbib}
\usepackage{subfiles}
\usepackage{tkz-berge}


\newcommand{\interval}[4]{\draw (#2, #1) -- (#3, #1); % Usage: \interval{height}{start}{end}{label}
\draw (#2, #1-0.11) -- (#2, #1+0.11); % draw left whisker
\draw (#3, #1-0.11) -- (#3, #1+0.11); % draw right whisker
\node[] at (#2-0.25, #1) {#4};
}

\newtheorem{prop}{Proposition}[section]
\newtheorem{thm}{Theorem}[section]
\newtheorem{lemma}{Lemma}[section]
\newtheorem{cor}{Corollary}[prop]

\theoremstyle{definition}
\newtheorem{definition}{Definition}

\theoremstyle{definition}
\newtheorem{required}{Problem}

\theoremstyle{definition}
\newtheorem{ex}{Example}


\setlength{\marginparwidth}{3.4cm}
%#########################################################

%To use symbols for footnotes
\renewcommand*{\thefootnote}{\fnsymbol{footnote}}
%To change footnotes back to numbers uncomment the following line
%\renewcommand*{\thefootnote}{\arabic{footnote}}

% Enable this command to adjust line spacing for inline math equations.
% \everymath{\displaystyle}

% _______ _____ _______ _      ______ 
%|__   __|_   _|__   __| |    |  ____|
%   | |    | |    | |  | |    | |__   
%   | |    | |    | |  | |    |  __|  
%   | |   _| |_   | |  | |____| |____ 
%   |_|  |_____|  |_|  |______|______|
%%%%%%%%%%%%%%%%%%%%%%%%%%%%%%%%%%%%%%%

\title{
\normalfont \normalsize 
\textsc{CSCI 3104 Spring 2023 \\ 
Instructors: Ryan Layer and Chandra Kanth Nagesh} \\
[10pt] 
\rule{\linewidth}{0.5pt} \\[6pt] 
\huge Homework 10 \\
\rule{\linewidth}{2pt}  \\[10pt]
}
\author{}
\date{}

\begin{document}

\definecolor {processblue}{cmyk}{0.96,0,0,0}
\definecolor{processred}{rgb}{200, 0, 0}
\definecolor{processgreen}{rgb}{0, 255, 0}
\DeclareGraphicsExtensions{.png}
\DeclareGraphicsExtensions{.gif}
\DeclareGraphicsExtensions{.jpg}

\maketitle


%%%%%%%%%%%%%%%%%%%%%%%%%
%%%%%%%%%%%%%%%%%%%%%%%%%%
%%%%%%%%%%FILL IN YOUR NAME%%%%%%%
%%%%%%%%%%AND STUDENT ID%%%%%%%%
%%%%%%%%%%%%%%%%%%%%%%%%%%
\noindent
Due Date \dotfill February 23, 2023 \\
Name \dotfill \textbf{Blake Raphael} \\
Student ID \dotfill \textbf{109752312} \\
Collaborators \dotfill \textbf{Alex Barry, Brody Cyphers, and Ben Kohav}

\tableofcontents

\section{Instructions}
 \begin{itemize}
	\item The solutions \textbf{should be typed}, using proper mathematical notation. We cannot accept hand-written solutions. \href{http://ece.uprm.edu/~caceros/latex/introduction.pdf}{Here's a short intro to \LaTeX.}
	\item You should submit your work through the \textbf{class Gradescope page} only (linked from Canvas). Please submit one PDF file, compiled using this \LaTeX \ template.
	\item You may not need a full page for your solutions; pagebreaks are there to help Gradescope automatically find where each problem is. Even if you do not attempt every problem, please submit this document with no fewer pages than the blank template (or Gradescope has issues with it).

	\item You are welcome and encouraged to collaborate with your classmates, as well as consult outside resources. You must \textbf{cite your sources in this document.} \textbf{Copying from any source is an Honor Code violation. Furthermore, all submissions must be in your own words and reflect your understanding of the material.} If there is any confusion about this policy, it is your responsibility to clarify before the due date. 

	\item Posting to \textbf{any} service including, but not limited to Chegg, Reddit, StackExchange, etc., for help on an assignment is a violation of the Honor Code.

	\item You \textbf{must} virtually sign the Honor Code (see Section \ref{HonorCode}). Failure to do so will result in your assignment not being graded.
\end{itemize}


\section{Honor Code (Make Sure to Virtually Sign)} \label{HonorCode}

\begin{itemize}
\item My submission is in my own words and reflects my understanding of the material.
\item Any collaborations and external sources have been clearly cited in this document.
\item I have not posted to external services including, but not limited to Chegg, Reddit, StackExchange, etc.
\item I have neither copied nor provided others solutions they can copy.
\end{itemize}

%\noindent In the specified region below, clearly indicate that you have upheld the Honor Code. Then type your name. 

\begin{proof}[Agreed (I agree to the above, Blake Raphael).]
%% Typing "I agree to the above," followed by your name is sufficient.
\end{proof}
\newpage
\section{Problem 1}
\begin{required}
Consider the following flow network, with no initial flow along the graph.
\begin{center}
	\begin {tikzpicture}[-latex ,auto ,node distance =2 cm and 3cm ,on grid ,
	semithick ,
	state/.style ={ circle ,top color =white , bottom color = processblue!20 ,
	draw,processblue , text=blue , minimum width =1 cm}]

	\node[state] (A) {$s$};
	\node[state] (B) [above right = of A] {$B$};
	\node[state] (C) [below right = of A] {$C$};
	\node[state] (D) [right = of B] {$D$};
	\node[state] (E) [right = of C] {$E$};
	\node[state] (F) [below right = of D] {$t$};
	
        \path (A) edge node[above] {$0 / 6$} (B);
        \path (A) edge node[above] {$0 / 7$} (C);
        \path (B) edge node[left] {$0 / 2$} (C);
        \path (B) edge node[above] {$0 / 3$} (D);
        \path (E) edge node[right] {$0 /  5$} (B);
        \path (C) edge node[above] {$0 / 10$} (E);
        \path (E) edge node[right] {$0 / 4$} (D);
        \path (D) edge node[above] {$0 / 9$} (F);
        \path (E) edge node[above] {$0 / 1$} (F);
	
	\end{tikzpicture}  
\end{center}


\noindent Do the following.
\begin{enumerate}[label=(\alph*)]
\subsection{Problem 1\ref{1a} (1 point)}
\item Consider the flow-augmenting path $s \to B \to D \to t$. Push as much flow through the flow-augmenting path and draw the updated flow network below. \label{1a}

\begin{proof}[Answer] Updated graph:\\ 
%If you wish, you may un-comment the tikzpicture below (remove the \begin{comment} and \end{comment} lines) and modify
%the edge weights. You may also hand-draw and embed an image

%Include an Image: \includegraphics{ImageFileName}
%Include an Image and Rotate 90 degree: \includegraphics[angle=90]{ImageFileName}
%Include an Image, Rotate by 180 degrees, and scale by 50\% \includegraphics[scale=0.5, angle=90]{ImageFileName}

\begin{center}
	\begin {tikzpicture}[-latex ,auto ,node distance =2 cm and 3cm ,on grid ,
	semithick ,
	state/.style ={ circle ,top color =white , bottom color = processblue!20 ,
	draw,processblue , text=blue , minimum width =1 cm}]

	\node[state] (A) {$s$};
	\node[state] (B) [above right = of A] {$B$};
	\node[state] (C) [below right = of A] {$C$};
	\node[state] (D) [right = of B] {$D$};
	\node[state] (E) [right = of C] {$E$};
	\node[state] (F) [below right = of D] {$t$};
	
        \path (A) edge node[above] {$3 / 6$} (B);
        \path (A) edge node[above] {$0 / 7$} (C);
        \path (B) edge node[left] {$0 / 2$} (C);
        \path (B) edge node[above] {$3 / 3$} (D);
        \path (E) edge node[right] {$0 /  5$} (B);
        \path (C) edge node[above] {$0 / 10$} (E);
        \path (E) edge node[right] {$0 / 4$} (D);
        \path (D) edge node[above] {$3 / 9$} (F);
        \path (E) edge node[above] {$0 / 1$} (F);
	
	\end{tikzpicture}  
\end{center}
\end{proof}




\newpage
\subsection{Problem 1\ref{1b} (1 point)}
\item \label{1b} Find a flow-augmenting path using the updated flow configuration from part (a). Then do the following: (i) clearly identify both the flow-augmenting path and the maximum amount of flow that can be pushed through said path; and then (ii) push as much flow through the flow-augmenting path and draw the updated flow network below.
\begin{proof}[Answer] The flow augmenting path is: $s \to C \to E \to t$ and the maximum amount of flow through this path is $1$.\\
%If you wish, you may un-comment the tikzpicture below (remove the \begin{comment} and \end{comment} lines) and modify
%the edge weights. You may also hand-draw and embed an image

%Include an Image: \includegraphics{ImageFileName}
%Include an Image and Rotate 90 degree: \includegraphics[angle=90]{ImageFileName}
%Include an Image, Rotate by 180 degrees, and scale by 50\% \includegraphics[scale=0.5, angle=90]{ImageFileName}



\begin{center}
	\begin {tikzpicture}[-latex ,auto ,node distance =2 cm and 3cm ,on grid ,
	semithick ,
	state/.style ={ circle ,top color =white , bottom color = processblue!20 ,
	draw,processblue , text=blue , minimum width =1 cm}]

	\node[state] (A) {$s$};
	\node[state] (B) [above right = of A] {$B$};
	\node[state] (C) [below right = of A] {$C$};
	\node[state] (D) [right = of B] {$D$};
	\node[state] (E) [right = of C] {$E$};
	\node[state] (F) [below right = of D] {$t$};
	
        \path (A) edge node[above] {$3 / 6$} (B);
        \path (A) edge node[above] {$1 / 7$} (C);
        \path (B) edge node[left] {$0 / 2$} (C);
        \path (B) edge node[above] {$3 / 3$} (D);
        \path (E) edge node[right] {$0 /  5$} (B);
        \path (C) edge node[above] {$1 / 10$} (E);
        \path (E) edge node[right] {$0 / 4$} (D);
        \path (D) edge node[above] {$3 / 9$} (F);
        \path (E) edge node[above] {$1 / 1$} (F);
	
	\end{tikzpicture}  
\end{center}
\end{proof}

\newpage
\subsection{Problem 1\ref{1c} (1 point)}
\item \label{1c} Find a flow-augmenting path using the updated flow configuration from part (b). Then do the following: (i) clearly identify both the flow-augmenting path and the maximum amount of flow that can be pushed through said path; and then (ii) push as much flow through the flow-augmenting path and draw the updated flow network below.
\begin{proof}[Answer] The flow augmenting path is: $s \to B \to C \to E \to D \to t$ and the maximum amount of flow through this path is $2$.\\
%If you wish, you may un-comment the tikzpicture below (remove the \begin{comment} and \end{comment} lines) and modify
%the edge weights. You may also hand-draw and embed an image

%Include an Image: \includegraphics{ImageFileName}
%Include an Image and Rotate 90 degree: \includegraphics[angle=90]{ImageFileName}
%Include an Image, Rotate by 180 degrees, and scale by 50\% \includegraphics[scale=0.5, angle=90]{ImageFileName}



\begin{center}
	\begin {tikzpicture}[-latex ,auto ,node distance =2 cm and 3cm ,on grid ,
	semithick ,
	state/.style ={ circle ,top color =white , bottom color = processblue!20 ,
	draw,processblue , text=blue , minimum width =1 cm}]

	\node[state] (A) {$s$};
	\node[state] (B) [above right = of A] {$B$};
	\node[state] (C) [below right = of A] {$C$};
	\node[state] (D) [right = of B] {$D$};
	\node[state] (E) [right = of C] {$E$};
	\node[state] (F) [below right = of D] {$t$};
	
        \path (A) edge node[above] {$5 / 6$} (B);
        \path (A) edge node[above] {$1 / 7$} (C);
        \path (B) edge node[left] {$2 / 2$} (C);
        \path (B) edge node[above] {$3 / 3$} (D);
        \path (E) edge node[right] {$0 /  5$} (B);
        \path (C) edge node[above] {$3 / 10$} (E);
        \path (E) edge node[right] {$2 / 4$} (D);
        \path (D) edge node[above] {$5 / 9$} (F);
        \path (E) edge node[above] {$1 / 1$} (F);	

	\end{tikzpicture}  
\end{center}
\end{proof}

\newpage
\subsection{Problem 1\ref{1d} (1 point)}
\item \label{1d} Using the flow configuration from part (\ref{1c}), finish executing the Ford--Fulkerson algorithm. Include the following here: (i) your flow network, reflecting the maximum-valued flow configuration you found, and (ii) the corresponding minimum capacity cut. There may be multiple minimum capacity cuts, but you should identify the one corresponding to your maximum-valued flow configuration. Then (iii) finally, compare the value of your flow to the capacity of the cut. \\

\noindent \textbf{Note:} You do \textbf{not} need to include the remaining steps of the Ford--Fulkerson algorithm. We will not check these steps when grading.
\begin{proof}[Answer] The minimum capacity cut is $s \to C \to E$ or $X = \{s, C, E\}$ because we can not push any more positive flow after that.\\
The value of the flow is $8$ and the value of the minimum capacity cut is $8$, which shows that we have found a minimum capacity cut according to the Max-Flow Min-Cut Theorem.\\
%If you wish, you may un-comment the tikzpicture below (remove the \begin{comment} and \end{comment} lines) and modify
%the edge weights. You may also hand-draw and embed an image

%Include an Image: \includegraphics{ImageFileName}
%Include an Image and Rotate 90 degree: \includegraphics[angle=90]{ImageFileName}
%Include an Image, Rotate by 180 degrees, and scale by 50\% \includegraphics[scale=0.5, angle=90]{ImageFileName}

\begin{center}
	\begin {tikzpicture}[-latex ,auto ,node distance =2 cm and 3cm ,on grid ,
	semithick ,
	state/.style ={ circle ,top color =white , bottom color = processblue!20 ,
	draw,processblue , text=blue , minimum width =1 cm}]

	\node[state] (A) {$s$};
	\node[state] (B) [above right = of A] {$B$};
	\node[state] (C) [below right = of A] {$C$};
	\node[state] (D) [right = of B] {$D$};
	\node[state] (E) [right = of C] {$E$};
	\node[state] (F) [below right = of D] {$t$};
	
        \path (A) edge node[above] {$5 / 6$} (B);
        \path (A) edge node[above] {$3 / 7$} (C);
        \path (B) edge node[left] {$2 / 2$} (C);
        \path (B) edge node[above] {$3 / 3$} (D);
        \path (E) edge node[right] {$0 /  5$} (B);
        \path (C) edge node[above] {$5 / 10$} (E);
        \path (E) edge node[right] {$4 / 4$} (D);
        \path (D) edge node[above] {$7 / 9$} (F);
        \path (E) edge node[above] {$1 / 1$} (F);
	
	\end{tikzpicture}  
\end{center}
\end{proof}

\end{enumerate}
\end{required}
\end{document} % NOTHING AFTER THIS LINE IS PART OF THE DOCUMENT
